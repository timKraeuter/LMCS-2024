\documentclass{lmcs} %%% last changed 2014-08-20

%% mandatory lists of keywords
\keywords{BPMN, Higher-order model transformation, Graph transformation, Model checking, Formalization}

%% read in additional TeX-packages or personal macros here:
%% e.g. \usepackage{tikz}
\usepackage{hyperref}
%%\input{myMacros.tex}
%% define non-standard environments BEYOND the ones already supplied
%% here, for example
\theoremstyle{plain}\newtheorem{satz}[thm]{Satz} %\crefname{satz}{Satz}{S\"atze}
%% Do NOT replace the proclamation environments lready provided by
%% your own.

\def\eg{{\em e.g.}}
\def\cf{{\em cf.}}

%% due to the dependence on amsart.cls, \begin{document} has to occur
%% BEFORE the title and author information:

\begin{document}

\title[Instructions]{Instructions for Authors\\How to prepare papers
  for LMCS using \texorpdfstring{\MakeLowercase{\texttt{lmcs.cls}}}{lmcs.cls}\rsuper*\\Version of
  2022-04-01}
\titlecomment{{\lsuper*}This paper is an extended version of \cite{krauterFormalizationAnalysisBPMN2023}}.
\thanks{thanks, optional.}	%optional

% affiliations are numbered automatically with a, b, c (see below)
% use the optional argument to indicate the affiliation(s) of each author
% omit the argument if there is only one author or only one affiliation
\author[A.~Name1]{Alice Name1}[a]
\author[B.~Name2]{Bob Name2}[a,b]
\author[J.~Name3]{Josiah S.~Carberry\lmcsorcid{0000-0002-1825-0097}}[a]

% affiliation 1 (automatically numbered a)
\address{University 1, address1}	%optional
% write emails for all authors having that affiliation
\email{name1@email1, name2@email1, name3@email1}  %optional

% affiliation 2 (automatically numbered b)
\address{University 2, address2}	%optional
\email{name2@email2}  %optional

%% etc.

%% required for running head on odd and even pages, use suitable
%% abbreviations in case of long titles and many authors:

%%%%%%%%%%%%%%%%%%%%%%%%%%%%%%%%%%%%%%%%%%%%%%%%%%%%%%%%%%%%%%%%%%%%%%%%%%%

%% the abstract has to PRECEDE the command \maketitle:
%% be sure not to issue the \maketitle command twice!

\begin{abstract}
  \noindent
The Business Process Modeling Notation (BPMN) is a widely used standard notation for defining intra- and inter-organizational workflows.
However, the informal description of the BPMN execution semantics leads to different interpretations of BPMN elements and difficulties in checking behavioral properties.
In this paper, we propose a formalization of the execution semantics of BPMN that, compared to existing approaches, covers more BPMN elements while facilitating property checking.
Our approach is based on a higher-order transformation from BPMN models to graph transformation systems.
As proof of concept, we have implemented our approach in an open-source web-based tool.
\end{abstract}

\maketitle
\section*{Introduction}\label{S:one}

% TODO: Add after review
% \section*{Acknowledgment}
%   \noindent We want to thank the anonymous reviewers for their valuable comments and helpful suggestions.
  
\bibliographystyle{alphaurl} 
\bibliography{bib}

\end{document}
