\documentclass{lmcs} %%% last changed 2014-08-20

%% mandatory lists of keywords
\keywords{BPMN, Higher-order model transformation, Graph transformation, Model checking, Formalization}

%% read in additional TeX-packages or personal macros here:
%% e.g. \usepackage{tikz}
\usepackage{hyperref}
%%\input{myMacros.tex}
%% define non-standard environments BEYOND the ones already supplied
%% here, for example
\theoremstyle{plain}\newtheorem{satz}[thm]{Satz} %\crefname{satz}{Satz}{S\"atze}
%% Do NOT replace the proclamation environments lready provided by
%% your own.

\def\eg{{\em e.g.}}
\def\cf{{\em cf.}}

%% due to the dependence on amsart.cls, \begin{document} has to occur
%% BEFORE the title and author information:

\begin{document}

\title[Formalization and analysis of BPMN using graph transformation systems]{Formalization and analysis of BPMN using graph transformation systems\rsuper*}
\titlecomment{{\lsuper*}This paper is an extended version of \cite{krauterFormalizationAnalysisBPMN2023}}
% \thanks{thanks, optional.}	%optional

\author[T.~Kr\"{a}uter]{Tim Kr\"{a}uter\lmcsorcid{0000-0003-1795-0611}}[a]
\author[A.~Rutle]{Adrian Rutle\lmcsorcid{0000-0002-4158-1644}}[a]
\author[H.~K\"{o}nig]{Harald K\"{o}nig\lmcsorcid{0000-0001-6304-6311}}[a,b]
\author[Y.~Lamo]{Yngve Lamo\lmcsorcid{0000-0001-9196-1779}}[a]

\address{Western Norway University of Applied Sciences, Bergen, Norway}
\email{tkra@hvl.no, aru@hvl.no, hkoe@hvl.no, yla@hvl.no}

\address{University of Applied Sciences, FHDW, Hanover, Germany}
\email{harald.koenig@fhdw.de}

%% etc.

%% required for running head on odd and even pages, use suitable
%% abbreviations in case of long titles and many authors:

%%%%%%%%%%%%%%%%%%%%%%%%%%%%%%%%%%%%%%%%%%%%%%%%%%%%%%%%%%%%%%%%%%%%%%%%%%%

%% the abstract has to PRECEDE the command \maketitle:
%% be sure not to issue the \maketitle command twice!

\begin{abstract}
  \noindent
The Business Process Modeling Notation (BPMN) is a widely used standard notation for defining intra- and inter-organizational workflows.
However, the informal description of the BPMN execution semantics leads to different interpretations of BPMN elements and difficulties in checking behavioral properties.
In this paper, we propose a formalization of the execution semantics of BPMN that, compared to existing approaches, covers more BPMN elements while facilitating property checking.
Our approach is based on a higher-order transformation from BPMN models to graph transformation systems.
As proof of concept, we have implemented our approach in an open-source web-based tool.
\end{abstract}

\maketitle
\section{Introduction}\label{S:one}

% TODO: Add after review
% \section*{Acknowledgment}
%   \noindent We want to thank the anonymous reviewers for their valuable comments and helpful suggestions.
  
\bibliographystyle{alphaurl} 
\bibliography{bib}

\end{document}
